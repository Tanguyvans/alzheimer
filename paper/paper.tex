\documentclass[conference]{IEEEtran}
\IEEEoverridecommandlockouts

\usepackage{cite}
\usepackage{amsmath,amssymb,amsfonts}
\usepackage{algorithmic}
\usepackage{graphicx}
\usepackage{textcomp}
\usepackage{xcolor}
\usepackage{booktabs}
\usepackage{multirow}

\def\BibTeX{{\rm B\kern-.05em{\sc i\kern-.025em b}\kern-.08em
    T\kern-.1667em\lower.7ex\hbox{E}\kern-.125emX}}

\begin{document}

\title{Multi-Cohort Alzheimer's Classification via MRI-Clinical Fusion}

\author{\IEEEauthorblockN{Tanguy Vansnick}
\IEEEauthorblockA{\textit{Faculty of Engineering - ILIA} \\
\textit{University of Mons}\\
Rue de Houdain 9, 7000 Mons, Belgium \\
tanguy.vansnick@umons.ac.be}
\and
\IEEEauthorblockN{Maxime Gloesener}
\IEEEauthorblockA{\textit{Faculty of Engineering - ILIA} \\
\textit{University of Mons}\\
Rue de Houdain 9, 7000 Mons, Belgium \\
maxime.gloesener@umons.ac.be}
\and
\IEEEauthorblockN{Otmane Amel}
\IEEEauthorblockA{\textit{Faculty of Engineering - ILIA} \\
\textit{University of Mons}\\
Rue de Houdain 9, 7000 Mons, Belgium \\
otmane.amel@umons.ac.be}
\and
\IEEEauthorblockN{Vito Tota}
\IEEEauthorblockA{\textit{Faculty of Engineering - ILIA} \\
\textit{University of Mons}\\
Rue de Houdain 9, 7000 Mons, Belgium \\
vito.tota@umons.ac.be}
\and
\IEEEauthorblockN{Sa\"id Mahmoudi}
\IEEEauthorblockA{\textit{Faculty of Engineering - ILIA} \\
\textit{University of Mons}\\
Rue de Houdain 9, 7000 Mons, Belgium \\
said.mahmoudi@umons.ac.be}
}

\maketitle

\begin{abstract}
Deep learning for Alzheimer's disease (AD) classification from MRI has shown promise, yet recent reviews reveal that fewer than 20\% of studies employ external validation, with data leakage inflating reported accuracies to 95-99\%. Studies with rigorous subject-level splitting achieve 66-90\%, establishing 80-85\% as the credible benchmark. We present an end-to-end multimodal framework combining Vision Transformer (ViT) with masked autoencoder pretraining for 3D MRI analysis and Feature Tokenizer Transformer (FT-Transformer) for clinical tabular data, integrated through bidirectional cross-attention. We validate on 6,065 subjects from three independent cohorts (ADNI, OASIS, NACC) using strict subject-level splitting and 5-fold cross-validation. Our approach achieves 93.3\% accuracy on the challenging AD-trajectory task (CN vs AD including MCI converters). Notably, cross-task evaluation reveals that trajectory-trained models achieve \textbf{96.1\% accuracy} and \textbf{92.1\% balanced accuracy} on stable AD detection, demonstrating that training on heterogeneous disease presentations improves generalization. Our results significantly exceed the 80-85\% credible benchmark while using only non-invasive measures and rigorous methodology.
\end{abstract}

\begin{IEEEkeywords}
Alzheimer's disease, multimodal fusion, Vision Transformer, deep learning, MRI classification
\end{IEEEkeywords}

\section{Introduction}

Alzheimer's disease (AD) is the most common cause of dementia, contributing to 60--70\% of the 57 million dementia cases worldwide \cite{who2023}. Early and accurate diagnosis is crucial for patient care, clinical trials enrollment, and treatment planning. Early diagnosis increasingly relies on biomarkers---CSF analysis requiring lumbar puncture, or amyloid PET imaging with radioactive tracers---which are impractical for routine screening. Structural MRI combined with clinical tabular data offers a practical alternative, capturing characteristic patterns of brain atrophy alongside demographic and cognitive assessments without invasive procedures or radiation exposure.

Deep learning approaches for AD classification have proliferated in recent years, with many studies reporting accuracies exceeding 95\%. However, a recent scoping review by Young et al. \cite{young2025} revealed that fewer than 20\% of published studies employ external validation, and methodological issues remain widespread. The primary issue is data leakage: when data from the same subject appears in both training and test sets, models learn to recognize individual brain morphology rather than disease-specific patterns.

Yagis et al. \cite{yagis2021} demonstrated that slice-level data splitting inflates accuracy by 25-55\%, with random labels achieving 96\% ``accuracy'' under improper splitting but only 50\% (as expected) with correct subject-level splitting. Studies employing rigorous methodology report accuracies in the 66-90\% range, with 80-85\% representing the credible benchmark for clinical translation \cite{young2025}.

Beyond methodological rigor, combining multiple data modalities has shown promise for improving AD classification. Clinical tabular data (demographics, cognitive assessments) provides complementary information to structural MRI. Prior multimodal approaches have achieved strong results, such as Qiu et al. \cite{qiu2020} who combined CNN-derived features with clinical variables including MMSE scores, achieving 96.8\% accuracy on ADNI but 79.2--93.2\% on external cohorts. However, no prior work has explored combining modern transformer architectures---specifically ViT for 3D MRI and FT-Transformer for tabular data---which represent the current state-of-the-art for their respective modalities, while excluding diagnostic scores to prevent data leakage.

In this work, we present a fully end-to-end multimodal deep learning framework that combines:
\begin{itemize}
    \item Vision Transformer (ViT) with masked autoencoder (MAE) pretraining for 3D MRI feature extraction
    \item Feature Tokenizer Transformer (FT-Transformer) for tabular clinical data
    \item Cross-attention fusion for learned modality interaction
\end{itemize}

Our contributions are: (1) the first end-to-end combination of 3D ViT and FT-Transformer for AD classification; (2) validation on 6,065 subjects from three independent cohorts (ADNI, OASIS, NACC) with strict subject-level splitting; (3) cross-task evaluation showing that training on heterogeneous AD presentations (including MCI converters) improves stable AD detection to 96.1\% accuracy; and (4) comprehensive ablation studies demonstrating the value of multimodal fusion.

\section{Related Work}

\subsection{Deep Learning for AD Classification}

Convolutional neural networks (CNNs) have dominated AD classification from MRI. Wen et al. \cite{wen2020} provided a comprehensive benchmark, showing that 3D CNNs achieve up to 86\% balanced accuracy with proper methodology. More recently, Vision Transformers (ViT) \cite{dosovitskiy2021} have been adapted for 3D medical imaging. Hoang et al. \cite{hoang2023} applied ViT to ADNI data, achieving 83\% for MCI-to-AD progression prediction. VECNN \cite{vecnn2024} combined ViT with CNNs, reaching 92.1\% on a single cohort.

\subsection{Methodological Concerns}

A critical issue in AD classification is data leakage. Young et al. \cite{young2025} found that fewer than 20\% of published studies employ external validation, with methodological issues widespread across the field. Yagis et al. \cite{yagis2021} demonstrated that improper splitting inflates accuracy by 25-55\%. Studies with rigorous subject-level splitting report 66-90\% accuracy, establishing 80-85\% as the credible benchmark.

\subsection{Transformer Architectures}

Vision Transformers (ViT) \cite{dosovitskiy2021} divide images into patches, treating each as a token for self-attention. Unlike CNNs with local receptive fields, ViT captures global relationships---critical for detecting distributed atrophy patterns in AD. Masked autoencoder (MAE) pretraining \cite{he2022} enables ViT to learn robust representations from unlabeled data by reconstructing masked patches.

For tabular data, the FT-Transformer \cite{gorishniy2021} embeds each feature as a separate token, enabling attention across features. This outperforms traditional MLPs on most benchmarks and achieves competitive results with gradient boosting, particularly when feature interactions matter.

\subsection{Multimodal Fusion}

Combining imaging with clinical data improves classification. Qiu et al. \cite{qiu2020} achieved 96.8\% accuracy on ADNI using an end-to-end FCN-MLP framework. In a larger follow-up study, Qiu et al. \cite{qiu2022} scaled to 8,916 subjects across 8 cohorts using CNN features with CatBoost for tabular data as separate models. No prior work has combined 3D ViT with FT-Transformer in an end-to-end multimodal framework for AD classification.

\section{Methods}

\subsection{Datasets}

We aggregated data from three publicly available cohorts:

\textbf{ADNI (Alzheimer's Disease Neuroimaging Initiative):} A longitudinal study tracking AD progression with well-curated MRI-clinical data pairs. We included 903 subjects (424 CN, 479 AD-trajectory).

\textbf{OASIS-3 (Open Access Series of Imaging Studies):} A longitudinal neuroimaging dataset from Washington University. We included 1,030 subjects (742 CN, 288 AD-trajectory).

\textbf{NACC (National Alzheimer's Coordinating Center):} The largest US Alzheimer's database, aggregating data from over 40 research centers. We included 4,132 subjects (3,581 CN, 551 AD-trajectory).

The AD-trajectory group includes both clinically diagnosed AD patients and MCI subjects who subsequently progressed to AD, representing individuals on the Alzheimer's disease continuum. The combined dataset comprises 6,065 subjects (4,747 CN, 1,318 AD-trajectory; 78.3\% vs 21.7\%). Table~\ref{tab:dataset} summarizes the dataset characteristics.

\begin{table}[htbp]
\caption{Dataset Composition}
\begin{center}
\begin{tabular}{lccccc}
\toprule
\textbf{Cohort} & \textbf{Total} & \textbf{CN} & \textbf{AD-traj} & \textbf{\% Total} \\
\midrule
ADNI & 903 & 424 & 479 & 14.9\% \\
OASIS & 1,030 & 742 & 288 & 17.0\% \\
NACC & 4,132 & 3,581 & 551 & 68.1\% \\
\midrule
\textbf{Total} & \textbf{6,065} & \textbf{4,747} & \textbf{1,318} & 100\% \\
\bottomrule
\end{tabular}
\label{tab:dataset}
\end{center}
\end{table}

\subsection{Data Splitting Strategy}

To prevent data leakage, we employed strict subject-level splitting:
\begin{itemize}
    \item Each subject appears in exactly one split (train/val/test)
    \item Only the first visit per subject is used (no longitudinal leakage)
    \item Splits are stratified by diagnosis and cohort
    \item 5-fold cross-validation with multiple random seeds
\end{itemize}

This approach follows recommendations from Wen et al.\ \cite{wen2020} and addresses methodological concerns raised by recent systematic reviews \cite{yagis2021}.

\subsection{MRI Preprocessing}

T1-weighted MRI scans underwent standardized preprocessing (Fig.~\ref{fig:preprocessing}):
\begin{enumerate}
    \item Conversion from DICOM to NIfTI format
    \item N4 bias field correction to remove intensity inhomogeneities
    \item Affine registration to MNI-152 template (1.75mm isotropic)
    \item Skull stripping using SynthStrip \cite{synthstrip}
    \item Resampling to 128$\times$128$\times$128 voxels
\end{enumerate}

\begin{figure}[htbp]
\centerline{\includegraphics[width=\columnwidth]{figures/preprocessing_pipeline.png}}
\caption{MRI preprocessing pipeline showing (left to right): original T1-weighted scan, N4 bias-corrected image, MNI-152 registered volume, and skull-stripped brain.}
\label{fig:preprocessing}
\end{figure}

\subsection{Clinical Features}

We extracted 16 clinical features common across all three cohorts:

\textbf{Demographics (4):} Age, sex, education years, marital status

\textbf{Neuropsychological tests (6):} Category fluency (animals), Trail Making Tests A/B, digit span forward/backward, Boston Naming Test

\textbf{Physical measurements (2):} Weight, BMI

\textbf{Medical history (4):} Alcohol use, smoking history, cardiovascular disease, neurological conditions

Missing values were imputed using median imputation and all features were z-score normalized. Importantly, diagnostic criteria scores (MMSE, CDR) were excluded to prevent data leakage.

\subsection{Model Architecture}

\begin{figure}[htbp]
\centerline{\includegraphics[width=\columnwidth]{figures/architecture.pdf}}
\caption{Multimodal fusion architecture combining Vision Transformer (ViT) for 3D MRI and FT-Transformer for clinical features with bidirectional cross-attention fusion.}
\label{fig:architecture}
\end{figure}

Our multimodal fusion architecture (Fig.~\ref{fig:architecture}) consists of three components:

\textbf{MRI Encoder:} We use a Vision Transformer (ViT-Base) adapted for 3D volumes, pretrained using masked autoencoder (MAE) reconstruction with 75\% masking ratio. Input volumes are divided into 16$\times$16$\times$16 patches, each embedded into 768-dimensional tokens. The encoder consists of 12 transformer blocks.

\textbf{Tabular Encoder:} We employ the FT-Transformer \cite{gorishniy2021}, which embeds each feature independently before processing through transformer layers. We use 3 transformer blocks with 64-dimensional embeddings and 4 attention heads, producing a 64-dimensional output.

\textbf{Cross-Modal Fusion:} Both modality representations are fused using bidirectional cross-attention with 8 heads and hidden dimension 512:
\begin{align}
f'_{mri} &= \text{CrossAttn}(Q{=}f_{mri}, K{=}f_{tab}, V{=}f_{tab}) \\
f'_{tab} &= \text{CrossAttn}(Q{=}f_{tab}, K{=}f_{mri}, V{=}f_{mri}) \\
f_{fused} &= \text{MLP}([f'_{mri}; f'_{tab}])
\end{align}
This enables each modality to attend to complementary information from the other. Auxiliary classifiers on each branch (weighted 0.3) encourage modality-specific learning.

\subsection{Training Details}

Models were trained for 100 epochs, saving the best checkpoint based on validation accuracy:
\begin{itemize}
    \item AdamW optimizer with weight decay 0.01
    \item Learning rate: 2$\times$10$^{-5}$ with cosine annealing
    \item 5 warmup epochs
    \item Layer-wise learning rate decay (0.75) for ViT fine-tuning
    \item Weighted cross-entropy loss with label smoothing (0.1)
    \item Batch size: 4 (due to 3D volume memory requirements)
    \item Test-time augmentation (8 augmentations)
\end{itemize}

Experiments used 5-fold cross-validation repeated with 5 random seeds (42, 123, 456, 789, 2024) to assess stability.

\section{Results}

\subsection{The Alzheimer's Disease Continuum}

Our classification groups established AD patients with MCI subjects who subsequently progressed to AD (MCI converters). This grouping is grounded in the 2018 NIA-AA Research Framework \cite{jack2018}, which defines AD as a biological continuum rather than a clinical syndrome---individuals with AD pathology are on the ``Alzheimer's continuum'' regardless of clinical stage.

Longitudinal neuroimaging studies provide strong evidence that MCI converters exhibit AD-like atrophy patterns detectable on structural MRI. Misra et al. \cite{misra2009} demonstrated using ADNI data that MCI converters show significantly higher hippocampal atrophy rates ($-3.5\%$/year) compared to stable MCI patients ($-2.2\%$/year), with baseline MRI achieving 81.5\% accuracy in predicting future conversion. Importantly, converters had already reached ``levels of widespread and significant brain atrophy at baseline,'' indicating that structural changes precede clinical diagnosis. Figure~\ref{fig:continuum} illustrates this progression from cognitively normal through MCI to AD dementia.

\begin{figure}[htbp]
\centerline{\includegraphics[width=\columnwidth]{figures/ad_continuum.png}}
\caption{The Alzheimer's disease continuum showing progressive brain atrophy from cognitively normal (CN) through mild cognitive impairment (MCI) to AD dementia. MCI converters exhibit intermediate atrophy patterns that predict future clinical progression.}
\label{fig:continuum}
\end{figure}

\subsection{Classification Performance}

All models were trained on the \textbf{CN vs AD-trajectory} task, where AD-trajectory includes both clinically diagnosed AD patients and MCI subjects who later progressed to AD. We evaluate on two test sets: (1) an in-domain AD-trajectory test set matching the training distribution, and (2) a cross-task \textbf{CN vs stable AD} test set containing only clinically confirmed AD patients without MCI converters.

Table~\ref{tab:results} presents our main results. On the in-domain AD-trajectory task, our multimodal model achieved 93.3\% accuracy and 87.9\% balanced accuracy. The gap reflects class imbalance (78\% CN vs 22\% AD-trajectory). Subgroup analysis reveals CN cases achieved 97.5\% accuracy, established AD 97.1\%, while MCI-to-AD converters were harder at 82.1\%---expected given their subtler atrophy patterns.

Remarkably, the same trajectory-trained model achieved \textbf{96.1\% accuracy} and \textbf{92.1\% balanced accuracy} on the stable AD test set, substantially exceeding its in-domain performance. This cross-task improvement demonstrates that training on heterogeneous presentations---including subtle early-stage MCI converters---forces the model to learn fine-grained discriminative features that generalize effectively to established disease with more pronounced atrophy.

\begin{table}[htbp]
\caption{Classification Results: Models Trained on CN vs AD-Trajectory}
\begin{center}
\begin{tabular}{lcccc}
\toprule
\textbf{Test Set} & \textbf{Acc} & \textbf{Bal. Acc} & \textbf{Sens} & \textbf{Spec} \\
\midrule
AD-trajectory (in-domain) & 93.3\% & 87.9\% & 78.3\% & 97.5\% \\
\textbf{Stable AD (cross-task)} & \textbf{96.1\%} & \textbf{92.1\%} & \textbf{86.2\%} & \textbf{98.1\%} \\
\bottomrule
\end{tabular}
\label{tab:results}
\end{center}
\end{table}

\subsection{Ablation Study}

To quantify the contribution of each modality, we evaluated single-modality baselines using identical training protocols (Table~\ref{tab:ablation}). The MRI-only model (ViT with MAE pretraining) achieved 84.1\% accuracy and 78.2\% balanced accuracy. The tabular-only model (FT-Transformer) achieved 83.5\% accuracy and 83.2\% balanced accuracy. Notably, the tabular model shows better balanced accuracy than MRI alone, suggesting clinical features provide strong signal for minority class (AD-trajectory) detection.

The multimodal fusion achieves substantial gains over both unimodal baselines: +7.5\% accuracy and +7.6\% balanced accuracy over MRI-only, and +8.1\% accuracy and +2.6\% balanced accuracy over tabular-only. This confirms that the modalities provide complementary information that cross-attention fusion effectively captures.

\begin{table}[htbp]
\caption{Ablation Study: Modality Contributions}
\begin{center}
\begin{tabular}{lcc}
\toprule
\textbf{Model} & \textbf{Accuracy} & \textbf{Balanced Acc.} \\
\midrule
MRI only (ViT) & 84.1\% & 78.2\% \\
Tabular only (FT-T) & 83.5\% & 83.2\% \\
\midrule
\textbf{Multimodal (Ours)} & \textbf{91.6\%} & \textbf{85.8\%} \\
\bottomrule
\end{tabular}
\label{tab:ablation}
\end{center}
\end{table}

\subsection{Comparison with Literature}

Table~\ref{tab:comparison} contextualizes our results within the broader literature. Studies with proper subject-level splitting report accuracies in the 66-90\% range \cite{young2025}, while those with potential data leakage claim 95-99\%.

\begin{table}[htbp]
\caption{Comparison with State-of-the-Art Methods}
\begin{center}
\footnotesize
\setlength{\tabcolsep}{3pt}
\begin{tabular}{@{}lcccrr@{}}
\toprule
\textbf{Study} & \textbf{Data} & \textbf{Coh.} & \textbf{N} & \textbf{Acc} & \textbf{AUC} \\
\midrule
\multicolumn{6}{@{}l}{\textit{Proper subject-level splitting:}} \\
\textbf{Ours (stable)} & MRI+Tab & 3 & 6,065 & \textbf{96.1\%} & --- \\
\textbf{Ours (traj)} & MRI+Tab & 3 & 6,065 & 93.3\% & --- \\
Qiu \cite{qiu2022} & MRI+Tab & 8 & 8,916 & ---$^*$ & .94--.97$^*$ \\
VECNN \cite{vecnn2024} & MRI & 1 & 818 & 94.1\% & --- \\
Hoang \cite{hoang2023} & MRI & 1 & 598 & 83\% & 0.87 \\
Wen \cite{wen2020} & MRI & 3 & 1k & 86\%$^\dagger$ & --- \\
\midrule
\multicolumn{6}{@{}l}{\textit{Potential data leakage:}} \\
Various \cite{young2025} & MRI & 1 & $<$1k & 95-99\% & --- \\
\bottomrule
\end{tabular}
\label{tab:comparison}
\end{center}
\vspace{-2mm}
{\scriptsize $^\dagger$Bal. acc.; $^*$Binary subtasks, not comparable}
\end{table}

Our 96.1\% accuracy on stable AD detection and 93.3\% on the challenging AD-trajectory task significantly exceed the 80-85\% ``credible benchmark'' established by Young et al. \cite{young2025} for methodologically rigorous studies. Key differentiators of our work:

\textbf{Scale and diversity:} With 6,065 subjects from 3 cohorts, ours is among the largest multi-cohort studies with proper methodology. Most ViT-based approaches use single cohorts with $<$1,000 subjects.

\textbf{End-to-end multimodal learning:} Unlike Qiu et al. \cite{qiu2022} who use separate CNN and CatBoost models, our approach learns joint MRI-clinical representations through cross-attention fusion.

\textbf{ViT improvement:} Compared to single-modality ViT approaches like Hoang et al. \cite{hoang2023} (83\%) and VECNN \cite{vecnn2024} (94.1\% on single cohort), our multimodal fusion achieves competitive performance while generalizing across cohorts.

\section{Discussion}

\subsection{Why Cross-Task Generalization Works}

Our most striking finding is that models trained on the heterogeneous AD-trajectory task (including MCI converters) achieve \textit{higher} accuracy on stable AD detection than on the training distribution itself. This counterintuitive result has a principled explanation: MCI converters exhibit subtle, early-stage atrophy patterns that are harder to distinguish from normal aging. Training on these challenging cases forces the model to learn fine-grained discriminative features---subtle hippocampal volume differences, distributed cortical thinning patterns, and nuanced feature interactions between imaging and clinical data.

When this model is evaluated on established AD cases with more pronounced atrophy, the learned features remain highly discriminative but the classification becomes easier. In contrast, a model trained only on obvious CN vs late-stage AD cases might learn coarse features that fail on subtle presentations. This suggests a general principle for medical imaging: training on heterogeneous, challenging cases can improve generalization to easier cases.

\subsection{Limitations}

Several limitations should be acknowledged:

\textbf{Class imbalance:} The 78\%/22\% CN/AD ratio affects sensitivity. Future work could explore oversampling or focal loss.

\textbf{Missing data:} Some tabular features have incomplete coverage across cohorts (23-100\%), handled via median imputation. Cohort-specific models could leverage richer feature sets.

\textbf{No external validation:} While we use three cohorts, they are mixed in training. True external validation (train on ADNI+OASIS, test on NACC) would provide stronger generalizability evidence.

\subsection{Future Directions}

Several extensions merit investigation. First, incorporating attention visualization could identify which brain regions and clinical features drive predictions, improving interpretability for clinical adoption. Second, extending to multi-class classification (CN vs MCI vs AD) would enable earlier intervention. Third, true external validation on held-out cohorts would strengthen generalizability claims. Finally, prospective clinical validation is needed before deployment as a screening tool.

\section{Conclusion}

We presented a multimodal deep learning framework for Alzheimer's disease classification combining Vision Transformer and Feature Tokenizer Transformer with cross-attention fusion. Validated on 6,065 subjects from three cohorts with strict subject-level splitting, we achieve 93.3\% accuracy on the challenging AD-trajectory task and 96.1\% on stable AD detection through cross-task generalization. Our cross-task evaluation reveals that training on heterogeneous disease presentations (including MCI converters) produces more discriminative features, improving generalization to established AD cases. These results significantly exceed the 80-85\% credible benchmark while maintaining methodological rigor.

\section*{Acknowledgment}

Data used in preparation of this article were obtained from the Alzheimer's Disease Neuroimaging Initiative (ADNI) database (adni.loni.usc.edu). As such, the investigators within the ADNI contributed to the design and implementation of ADNI and/or provided data but did not participate in analysis or writing of this report.

Data were provided in part by OASIS-3: Principal Investigators: T. Benzinger, D. Marcus, J. Morris; NIH P30 AG066444, P50 AG00561, P01 AG026276, P01 AG003991.

The NACC database is funded by NIA/NIH Grant U24 AG072122. NACC data are contributed by the NIA-funded ADRCs.

\begin{thebibliography}{00}
\bibitem{who2023} World Health Organization, ``Dementia,'' 2023. [Online]. Available: https://www.who.int/health-topics/dementia

\bibitem{young2025} V. M. Young, S. Gates, L. Y. Garcia, and A. Salardini, ``Data leakage in deep learning for Alzheimer's disease diagnosis: A scoping review of methodological rigor and performance inflation,'' \textit{Diagnostics}, vol. 15, no. 18, p. 2348, 2025.

\bibitem{yagis2021} E. Yagis, A. G. S. De Herrera, and L. Citi, ``Effect of data leakage in brain MRI classification using 2D convolutional neural networks,'' \textit{Sci. Rep.}, vol. 11, no. 22544, 2021.

\bibitem{qiu2020} S. Qiu et al., ``Development and validation of an interpretable deep learning framework for Alzheimer's disease classification,'' \textit{Brain}, vol. 143, no. 6, pp. 1920--1933, 2020.

\bibitem{wen2020} J. Wen et al., ``Convolutional neural networks for classification of Alzheimer's disease: Overview and reproducible evaluation,'' \textit{Med. Image Anal.}, vol. 63, p. 101694, 2020.

\bibitem{dosovitskiy2021} A. Dosovitskiy et al., ``An image is worth 16x16 words: Transformers for image recognition at scale,'' in \textit{Proc. ICLR}, 2021.

\bibitem{hoang2023} G. M. Hoang et al., ``Vision Transformers for predicting mild cognitive impairment to Alzheimer's disease progression using ADNI data,'' \textit{Front. Aging Neurosci.}, vol. 15, p. 1102730, 2023.

\bibitem{vecnn2024} Z. Zhao et al., ``Vision transformer-equipped convolutional neural networks for automated Alzheimer's disease diagnosis using 3D MRI scans,'' \textit{Front. Neurol.}, vol. 15, p. 1490829, 2024.

\bibitem{he2022} K. He, X. Chen, S. Xie, Y. Li, P. Dollar, and R. Girshick, ``Masked autoencoders are scalable vision learners,'' in \textit{Proc. CVPR}, 2022.

\bibitem{gorishniy2021} Y. Gorishniy, I. Rubachev, V. Khrulkov, and A. Babenko, ``Revisiting deep learning models for tabular data,'' in \textit{Proc. NeurIPS}, 2021.

\bibitem{qiu2022} S. Qiu et al., ``Multimodal deep learning for Alzheimer's disease dementia assessment,'' \textit{Nat. Commun.}, vol. 13, p. 3404, 2022.

\bibitem{synthstrip} A. Hoopes, J. S. Mora, A. V. Dalca, B. Fischl, and M. Hoffmann, ``SynthStrip: Skull-stripping for any brain image,'' \textit{NeuroImage}, vol. 260, p. 119474, 2022.

\bibitem{jack2018} C. R. Jack Jr. et al., ``NIA-AA Research Framework: Toward a biological definition of Alzheimer's disease,'' \textit{Alzheimer's Dement.}, vol. 14, no. 4, pp. 535--562, 2018.

\bibitem{misra2009} C. Misra, Y. Fan, and C. Davatzikos, ``Baseline and longitudinal patterns of brain atrophy in MCI patients, and their use in prediction of short-term conversion to AD: Results from ADNI,'' \textit{NeuroImage}, vol. 44, no. 4, pp. 1415--1422, 2009.

\end{thebibliography}

\end{document}
